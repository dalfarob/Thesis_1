\section{\textbf{Metodolog\'ia}}
\subsection{Dise\~no de Experimentos}
Para describir el planeamiento pre-experimental para el dise\~no de experimentos de este trabajo, (con la informaci\'on disponible hasta el momento), se usan los \textit{lineamientos} desarrollados en el libro de Douglas C. Montgomery [2]. El esquema del procedimiento recomendado en los lineamientos para esta etapa incluye lo siguiente:
\begin{itemize}
\item [1.] \textbf{Reconocimiento y definici\'on del problema:} consiste en desarrollar una declaraci\'on clara y sencilla del problema. Una clara definici\'on del problema, normalmente contribuye substancialmente a una mejor comprensi\'on del fen\'omeno que esta siendo estudiado y a la soluci\'on final de dicho problema.
\item [2.] \textbf{Selecci\'on de factores, niveles y rangos:} consiste en enumerar todos los posibles factores que pueden influenciar el experimento. Incluye tanto los factores de dise\~no potencial (los que potencialmente se podr\'ian querer modificar en los experimentos) y los factores perturbadores (los que no se quieren estudiar en el contexto del experimento). Tambi\'en se deben seleccionar los rangos sobre los que var\'ian los distintos factores y los niveles espec\'ificos sobre los que se aplicar\'an las iteraciones del experimento.
\item [3.] \textbf{Selecci\'on de la variable de respuesta:} debe proveer informaci\'on \'util sobre el fen\'omeno que esta siendo estudiado.
\item [4] \textbf{Selecci\'on del dise\~no de experimental:} se refiere a aspectos claves del experimento tales como el tama\~no de la muestra, la selecci\'on del orden adecuado para la ejecuci\'on de los intentos experimentales y la decisi\'on de bloquear o no algunas de las restriciones de aleatoriedad en la pruebas.
\item [5] \textbf{Llevar a cabo el experimiento:} en esta etapa, es de vital importancia monitorear el proceso cuidadosamente para asegurar la correcta ejecuci\'on del experimento con respecto a lo planeado.
\end{itemize}
\subsubsection{Declaraci\'on del Problema}
\subsubsection{Factores y Niveles}
\subsubsection{Variables de Respuesta}
\subsubsection{Recolecci\'on de Datos}
\subsubsection{An\'alisis de Varianza}
\subsection{Ambiente de Desarrollo}
