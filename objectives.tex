\section{\textbf{Objetivo General}}
Estudiar el nivel de exactitud obtenido al utilizar \textit{Cubic Spline Interpolation} como medida de distancia utilizada en el descubrimiento de reglas significativas en series temporales complejas y en presencia de ruido.
\section{\textbf{Objetivos Espec\'ificos}}
Los objetivos espec\'ificos de este proyecto son los siguientes:
\begin{itemize}
\item [1.] Proponer el uso de \textit{Cubic Spline Interpolation} como medida de distancia utilizada en los algoritmos creados por Mohammad Shokoohi-Yekta y colaboradores (\textit{\textbf{\enquote{Rule Bit Saves}}} y \textit{\textbf{\enquote{Find Antecedent Candidates}}}), para el descubrimiento de reglas significativas en series temporales complejas y en presencia de ruido.
	\item [2.] Realizar un an\'alisis comparativo del nivel de exactitud obtenido al utilizar diferentes medidas de distancia en los algoritmos mencionados.
\item [3.] Explicar los resultados obtenidos en el objetivo espec\'ifico anterior, con el prop\'osito de aprobar o rechazar la hip\'otesis planteada.
\end{itemize}