\section{\textbf{Propuesta de Proyecto}}
\subsection{Planteamiento del Problema}
El c\'alculo de la similitud en series temporales ha sido un tema muy estudiado en la \'ultima d\'ecada \cite{rulediscovery}. La precisi\'on, la velocidad de c\'omputo y la tolerancia al ruido, son factores claves (especialmente en conjuntos de datos grandes y complejos) a la hora de elegir una medida de distancia robusta para comparar dos series de tiempo de largo n \cite{multidimensional}.\\\\
En la literatura, la medida de distancia m\'as utilizada para comparar series temporarles es sin duda la distancia \textit{Euclidiana} (o alguna de sus variaciones). Dicha medida es principalmente aplicada en el descubrimiento y la comparaci\'on de motivos o patrones en series temporales \cite{motifs}\cite{patterns}.\\\\
Existen pruebas empiricas fiables, que demuestran que la distancia Euclidiana es muy competitiva e incluso superior a medidas mucho m\'as complejas en una amplia variedad de dominios, particularmente cuando el conjunto de datos se vuelve cada vez m\'as grande \cite{distancecomparison}\cite{timewarpingindexing}.\\\\
Sin embargo, algunos aportes m\'as recientes al estado del arte indican que las medidas de distancia conocidas como \textit{Dynamic Time Warping}, en una o varias dimensiones, puede incluso comportarse de forma m\'as robusta que la distancia \textit{Euclidiana} \cite{keogh}. Este argumento se soporta principalmente en la sensibilidad conocida que presenta la distancia \textit{Euclidiana} ante la presencia de ruido o pequen\~nas distorciones observadas al comparar dos series temporales desfasadas con respecto al eje de tiempo \cite{DTWcubicsplineinterpolation}.\\\\
Estimar el grado de ruido en un conjunto de datos es una tarea dif\'icil, lo que s\'i es seguro, es que la presencia de ruido en series temporales es inevitable o pr\'acticamente inherente \cite{noise}.\\\\
\subsection{Propuesta del Proyecto}
Apoyados en la premisa anterior, el proyecto pretende estudiar el nivel de acierto obtenido como resultado del descubrimiento de \textit{\enquote{reglas significativas}} en series de tiempo, utilizando \textit{Cubic Spline Interpolation} como una medida alternativia de distancia aparentemente superior a la distancia \textit{Euclidiana}, principalmente ante la presencia de distorciones en el conjunto de datos.\\\\
El proyecto se enfoca en remplazar la distancia Euclidiana y probar que la utilizaci\'on de otras medidas de distancia (particularmente el uso de \textit{Cubic Spline Interpolation}) pueden ser mucho m\'as tolerantes al ruido y a\'un as\'i garantizar al menos el mismo nivel de acierto en el hallazgo de reglas significativas en series temporales.
\subsection{Trabajos Relacionados}
\subsection{Hip\'otesis}
Con base en la definici\'on del problema y en la propuesta de proyecto, se define la siguiente hip\'otesis:\\\\
\textbf{\textit{El uso de la medida de distancia \textit{Cubic Spline Interpolation}, mejora el nivel de exactitud en los algoritmos \textit{\textbf{\enquote{Rule Bit Saves}}} y \textit{\textbf{\enquote{Find Antecedent Candidates}}} propuestos por Mohammad Shokoohi-Yekta y colaboradores, en el hallazgo de reglas significativas en series de tiempo complejas y en presencia de ruido.}} 
\subsection{M\'etricas}
El an\'alisis comparativo de los niveles de exactitud obtenidos a partir de la ejecuci\'on de los algoritmos seg\'un la distancia utilizada, requerir\'a de las siguientes m\'etricas:
\begin{itemize}
\item \textbf{Exactitud (Q):}
\begin{equation}
 \frac{Total\_Aciertos} {Total\_Predicciones}
\end{equation}
\end{itemize}
En el caso m\'as general, se utilizar\'a inicialmente la distancia Euclidiana entre la parte consecuente predicha y las \textit{\textbf{F}} ubicaciones halladas desde donde la regla fue disparada, un valor denotado como \textit{\textbf{\enquote{Ferror}}}, tambi\'en conocido como la \textit{media cuadr\'atica}.\\\\
Sobre el mismo conjunto de prueba, y mediante el uso del mismo segmento consecuente de la regla, se disparar\'a aleatoriamente \textit{\textbf{F}} veces y se medir\'a la distancia \textit{Euclidiana} (Cubic Spline Interpolation y otras), entre el segmento consecuente predicha y la ubicaciones aleatorias F.\\\\
Ese valor ser\'a denotado como \textit{\textbf{Rerror}} (el cual, se promediar\'a entre 1000 ejecuciones aleatorias).\\\\
En resumen, la medida de calidad reportada puede definirse como: 
\begin{equation} 
Q = \frac{Ferror}{Rerror} 
\end{equation}
Los valores cercanos a uno, sugieren que las reglas a prueba, no se consideran mejores que las encontradas en la estimaci\'on aleatoria. Los valores significativamente menores a uno, indican que la regla en efecto encuentra una estructura verdadera en los datos. En la mayor\'ia de los experimientos se utilizar\'a un retraso m\'aximo (\textit{\textbf{maxlag}}) de 0 .
\subsection{Justificaci\'on del Proyecto}