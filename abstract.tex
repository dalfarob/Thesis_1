The ability to make short or long term predictions is at the heart of much of science.
In the last decade, the data science community have been highly interested in foretelling about future events by using data mining techniques to find out meaningful rules from different data types, including \textit{Time Series}. Short-term predictions based on \textit{the shape} of meaningful rules may have a vast number of applications in real life. The discovery of \textit{meaningful} rules can be only achieved as a result of algorithms equipped with a robust distance measure, capable to compute precise and noise intolerant similarity results between time series elements. In this work, we believe that \textit{Cubic Spline Interpolation} can be utilized as an efficient distance measure to perform the similarity computation attached into two main algorithms: 1- \textit{\enquote{Rule Bit Saves}} and 2- \textit{\enquote{Find Antecedent Candidates}}, which were proposed by Mohammad Shokoohi-Yekta et al, to discover meaningful rules from complex time series in presence of noise.