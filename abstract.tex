\subsection*{\textbf{Abstract}}
The ability to make short or long term predictions is at the heart of much of science. In the last decade, the data science community have been highly interested in foretelling about real life events by using data mining techniques to find out meaningful rules from different data types, including \textit{Time Series}. Short-term predictions based on \textit{the shape} of meaningful rules might lead to a vast number of applications. The discovery of \textit{meaningful} rules can only be achieved as a result of algorithms equipped with a robust and accurate distance measure, capable to deal with noise in order to get the best possible similarity results between the elements of the time series. In this work, we believe that \textit{Cubic Spline Interpolation} can be used as an efficient distance measure, to carry out the similarity computation in two specific algorithms: 1- \textit{\enquote{Rule Bit Saves}} and 2- \textit{\enquote{Find Antecedent Candidates}}, which were proposed by Mohammad Shokoohi-Yekta et al, to discover meaningful rules from complex time series, in presence of noise.\par
\subsection*{\textbf{Resumen}}
La capacidad de hacer predicciones de largo o de corto plazo, ha estado siempre en el coraz\'on de la ciencia. En la \'ultima d\'ecada, la comunidad cient\'ifica de datos ha mostrado un gran inter\'es en vaticinar eventos de la vida real, mediante el uso de t\'ecnicas de miner\'ia de datos, para hallar reglas significaticas a partir de diversos tipos de datos, incluyendo el an\'alisis de series temporales. Las predicciones basadas en \textit{la forma} de la regla significativa puede dar lugar a un amplio n\'umero de aplicaciones. El descubrimiento de reglas significativas solo puede alcanzarse como resultado del uso de algoritmos equipados con una medida de distancia robusta y precisa, capaz de lidiar con el ruido, para obtener los mejores resultados de similitud posibles entre los elementos de las series temporales. En este trabajo, creemos que \textit{Cubic Spline Interpolation} puede ser usado como una medida de distancia efficiente, para llevar a cabo el c\'omputo de la similitud en dos algoritmos espec\'ificos: 1- \textit{\enquote{Rule Bit Saves}} Y 2- \textit{\enquote{Find Antecedent Candidates}}, los cuales, fueron propuestos por Mohammad Shokoohi-Yekta et al, para el descubrimiento de reglas significativas en series temporales complejas, en presencia de ruido.