\section{\textbf{Introducci\'on}}
La habilidad de hacer predicciones acerca de acontecimientos o eventos de la vida real ha sido siempre un tema de gran inter\'es para la ciencia.\par 
En la \'ultima d\'ecada, la comunidad de miner\'ia de datos se ha interesado vehementemente en el hallazgo de patrones o reglas que puedan ser \'utiles en la predicci\'on de eventos de corto y de largo plazo \cite{main}.\par
La mayor\'ia de trabajos de investigaci\'on recientes, orientados en la predicci\'on de eventos de corto plazo mediante series temporales, se han enfocado principalmente en el an\'alisis de los \textit{\textbf{valores actuales}} del flujo de datos \cite{rulediscovery}\cite{subsequencematching}. Sin embargo, en una basta cantidad de casos, el an\'alisis de los valores actuales es irrelevante; en su lugar, la \textit{\textbf{forma}} actual del patr\'on o la regla motivo y la detecci\'on oportuna en el flujo de datos, pueden ayudar a anticipar la ocurrencia de eventos futuros con mayor precisi\'on \cite{main}.\par
Este trabajo de investigaci\'on tiene como objetivo principal, la implementaci\'on de la medida de distancia llamada \textit{Cubic Spline Interpolation} en los algoritmos \textit{\textbf{\enquote{Rule Bit Saves}}} y \textit{\textbf{\enquote{Find Antecedent Candidates}}}, utilizados respectivamente en el hallazgo y la detecci\'on de reglas significativas, para llevar a cabo predicciones de corto plazo, sobre series temporales complejas y en presencia de ruido.\par
Las predicciones de corto plazo sobre series temporales han tenido un auge importante, su aplicaci\'on y alcance se ha diversificado considerablemente. Las predicciones de corto plazo sobre texto durante las pulsaciones del teclado, predicciones sobre consultas de base de datos \cite{type}, predicciones sobre intervenciones m\'edicas \cite{medical}, son solo algunos ejemplos de predicciones sobre objectos discretos.\par
Recientemente, ha surgido una reto a\'un mayor; se requiere un mayor poder predictivo, lo que implica necesariamente la implementaci\'on de algoritmos de predicci\'on mucho m\'as precisos, m\'as veloces y que puedan hallar patrones sobre conjuntos de datos mucho m\'as grandes y complejos \cite{robotics}. Por ejemplo, el radar Doppler utilizado en las \'ultimas dos d\'ecadas para la detecci\'on de tornados, ha incrementado el tiempo promedio de alerta de 5.3 a 9.5 minutos, salvando un incontable n\'umero de vidas humanas a\~no con a\~no. Sin embargo, a\'un se reportan alrededor de un 26\% de tornados que no pueden predecirse mediante el uso de la tecnolog\'ia existente \cite{weatherforcasting}. McGovern et al. argumentan en \cite{weatherprediction}, que las nuevas mejoras no vendr\'an necesariamente de sensores m\'as sofisticados, sino, de algoritmos de predicci\'on a\'un no inventados o algoritmos existentes a\'un no depurados, capaces de examinar series temporales complejas, para hallar reglas predictivas mucho m\'as precisas y fiables.\par
La presente propuesta de investigaci\'on se encuentra distribu\'ida de la siguiente manera: inicialmente, en la primera secci\'on, se presenta la introducci\'on del documento; posteriormente, en la secci\'on dos, se desarroll\'a el marco te\'orico como un grupo de ideas y conceptos fundamentales, que tienen como objetivo principal, guiar e involucrar al lector en el contexto de esta propuesta de investigaci\'on. En la secci\'on tres, se exponen los detalles m\'as importantes de la propuesta de investigaci\'on, tales como el planteamiento del problema, la hip\'otesis, las m\'etricas utilizadas y del por qu\'e la importancia de llevar a cabo esta investigaci\'on. El objetivo general y los objetivos espec\'ificos de esta propuesta, se ofrecen en las secciones cuatro y cinco respectivamente. El alcance de la investigaci\'on, ser\'an puntualmente detalladas en la secci\'on seis, mientras que los entregables ser\'an enlistados en la secci\'on siete. Por otra parte, en la secci\'on ocho, se describir\'a el dise\~no experimental y el ambiente de desarrollo que le dar\'an forma a la metodolog\'ia utilizada. Finalmente, en la secci\'on nueve, se presenta el cronograma de actividades establecido, para la llevar a cabo la realizaci\'on de este proyecto de investigaci\'on.