\section{\textbf{Alcance de la Investigaci\'on}}
El alcance de esta propuesta de investigaci\'on se enfoca espec\'ificamente en estudiar el efecto de utilizar \textit{\enquote{Cubic Spline Interpolation}} como medida de distancia en los algoritmos \textit{\textbf{\enquote{Rule Bit Saves}}} y \textit{\textbf{\enquote{Find Antecedent Candidates}}} propuestos por Mohammad Shokoohi-Yekta y colaboradores, utilizados para el hallazgo de reglas significativas en series de tiempo complejas y en presencia de ruido.\par
El dise\~no experimental incluye tambi\'en la comparaci\'on y el an\'alisis de cinco versiones de ambos algoritmos. Cada versi\'on incorpora la implementaci\'on de cinco medidas de distancia distintas: 1- Euclideana, 2- Swale, 3- Spade, 4- EPR y 5- Cubic Spline Interpolation.\par
Como resultado de esta investigaci\'on, se entregar\'an los siguientes productos:
\begin{itemize}
\item La implementaci\'on de las cinco versiones de ambos algoritmos para cada una de las medidas de distancia.
\item Programas auxiliares para la ejecuci\'on y el control del entregable anterior.
\item An\'alisis estad\'istico que contraste los resultados de los experimentos para aceptar o rechazar la hip\'otesis.
\item Un art\'iculo cient\'ifico que se entregar\'a al comite editorial de alguna revista o conferencia, con miras a su publicaci\'on.
\end{itemize}
\par
Es necesario delimitar esta investigaci\'on por motivos de tiempo y extensi\'on. Es por ello que a continuaci\'on se detallan las siguientes excepciones:
\begin{itemize}
\item No se tomar\'an en cuenta otras medidas de distancia.
\item No se utilizar\'a ning\'un otro algoritmo para la identificaci\'on de reglas \textit{motif}.
\item No se utilizar\'a ning\'un otro algoritmo para la detecci\'on de segmentos antecedentes de una potencial regla significativa.
\item Cualquier otro resultado, documento, software o producci\'on que no se encuentren contemplados en los entregables, no ser\'a considerado como parte del alcance de este proyecto.
\end{itemize}
En resumen, debe considerarse el hecho de que el objetivo principal de esta investigaci\'on, se enfoca principalmente en la implemetaci\'on de las medidas de distancia sobre ambos algoritmos, para su posterior an\'alisis comparativo, aplicado a cada unos de los cinco diferentes conjuntos de datos; utilizando \textit{\enquote{Cubic Spline Interpolation}} como el elemento m\'as importante de la hip\'otesis planteada.